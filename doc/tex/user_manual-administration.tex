This Chapter will explain in detail how to manage the users for your OCEMR
system. However, before we get into the details of doing that, we need to
go over a few very important things.

\section{Precautions}

Managing users will be done through the application’s admin interface.
The admin interface can be extremely powerful and helpful, but also
dangerous. Please be careful while using it and follow the instructions
outlined here. Contact your system administrator or help desk person if
you have any questions.

\subsection{Relational Database Warning}

  OCEMR has, at it’s heart, a relational database that stores information
in an efficient and streamlined manner. When one piece of information in
the database refers to another, like when a Diagnosis is added for a Patient,
rather than store a copy of the patient’s name, the database store a reference
to that patient record. In most cases this is extremely helpful. Like if we
realize two visit’s in that patient’s name is spelled wrong. We edit the
patient’s name once and everywhere that record is referenced after that the
new name will appear.

  However, this can have drastic complications. Say, for instance, we delete
the patient we were talking about in the previous example. That patient is
referenced in all sorts of records throughout the system. All the notes,
symptoms, medication, diagnoses that were added refer back to that patient
record. Without the patient record, all those other records are
useless\footnote{ in computer term they are ‘orphans’ because their
‘parents’ have disappeared } and will be deleted along with it.

  In the case of the patient record, this makes sense, but consider that
most of the records in the system link back to a user. Thus if you delete
a user, everything record in the system that that user added would also be
deleted. \textbf{BOTTOM LINE IS}:

\textbf{NEVER DELETE A USER FROM THE SYSTEM!} (Just disable the user instead.)

\section{Navigating to the Admin Interface}

The admin interface is not straight forward to get to. There is
currently\footnote{ v0.4.0 plans are to display a link for users
that have access in an upcoming release. } no link from the main
interface. You need to modify the link in your browser.

\begin{enumerate}
\item Log into the system.
\item in the Address Bar of your browser you will see something like: \url{http://ocemr.server.com/patient_queue/}.
\item change it to: \url{http://ocemr.server.com/admin/}
\end{enumerate}
